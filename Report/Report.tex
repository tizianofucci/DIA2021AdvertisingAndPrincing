\documentclass[12pt,a4paper]{report}

%Set language
\usepackage[english]{babel}
\usepackage{enumerate}

% To import and adjust images
\usepackage{graphicx}
\usepackage[export]{adjustbox}
\usepackage[center]{caption}
\usepackage{subcaption}
\usepackage{float}
\usepackage{tabularx}
\usepackage{lipsum}
\usepackage{caption}
\usepackage{eurosym}

% To use monospaced font
\usepackage{courier}

% To build a clickable Toc
\usepackage{color} %May be necessary if you want to color links
\usepackage{hyperref}
\hypersetup{
    colorlinks=true, %set true if you want colored links
    linktoc=all,     %set to all if you want both sections and subsections linked
    linkcolor=black,  %choose some color if you want links to stand out
    urlcolor = black
}


%To load PoLitecnico's logo
\usepackage{titling}

% Command to hide subsections in the Toc
\setcounter{tocdepth}{1}

% I don't like dots in the Toc
\usepackage{tocloft}
\renewcommand{\cftdot}{}

%To improve the tables
\usepackage[table]{xcolor}

%To break line inside tables
%\usepackage[utf8]{inputenc}
%\usepackage{fourier} 
%\usepackage{array}
\usepackage{makecell}
%\renewcommand\theadalign{bc}
%\renewcommand\theadfont{\bfseries}
\renewcommand\theadgape{\Gape[4pt]}

% Path relative to the .tex file containing the \includegraphics command
\graphicspath{ {./images/} }

% To change the ToC title
\addto\captionsenglish{ \renewcommand {\contentsname} {Table of contents}}
 
%logo
\pretitle{
	 \begin{center}
	 \LARGE
	 \includegraphics[width = 0.6\textwidth]{logo}\\[\bigskipamount]
}
\posttitle{\end{center}}

% Here we go
\title{Data Intelligence Applications Homework}
\author{D'Amato Francesco, \\
	Frantuma Elia - 10567359 - 945729, \\
	Fucci Tiziano - 10524029 - 946638}
\date{A.Y. 2020/2021}

\begin{document}
	\maketitle
	%Index
	\tableofcontents
	\chapter{Introduction}
		\section{Scenario}
			Consider the scenario in which advertisement is used to attract users on an ecommerce website and the users, after the purchase of the first unit of a consumable item, will buy additional units of the same item in future. The goal is to find the best joint bidding and pricing strategy taking into account future purchases.

\begin{figure}[H]
\centering
  \includegraphics[scale = 0.3, center]{image0}
  \caption{Advertising example}
\end{figure}

\section{The product}
The product we have chosen to simulate this advertising scenario is an energy drink. As we will say later, the first unit of product comes with a "dash button", to encourage the customer to buy it again and simulate the re-buy process.
\begin{figure}[H]
\centering
  \includegraphics[scale = 0.7, center]{redbull-dash}
  \caption{The sold product}
\end{figure}
	%end of first chapter

	\chapter{Environment}		
In this section we give a precise definition of the customer classes and their features, cost functions and distribution probabilities on which the model is based.
		\section{Customer classes}
In the environment model we have three customer classes: C1, C2 and C3.
They represent customers with different needs, age and tastes and thus different interest in buying the product.
			\subsection{Class 1: the sportsman}
The first class is composed by people who play sports or train regularly. They are interested in buying the energy drink to improve their performance. 


Their demand curve is: [inserire formula]

and the probability to buy again the product after the first time is described by: [inserire formula e grafici]
			\subsection{Class 2: the programmer}
Programmers need to stay focused for a long time while at work, so they are interested in buying the product to work better and avoid introducing bugs in the code.

Their demand curve is: [inserire formula]

and the probability to buy again the product after the first time is described by: [inserire formula e grafici]
			\subsection{Class 3: the retired man}
The average member of this class buys the product just to enjoy its taste.
His demand curve is: [inserire formula]

and the probability to buy again the product after the first time is described by: [inserire formula e grafici]

		\section{Advertising}
			\subsection{Auctions}
In our model, we make the hypothesis of a GSP (Generalized Second Price) auction mechanism:
	\[p_a= \frac{q_a}{q_{a+1}}v_{a+1} \left( \leq\frac{q_a}{q_a}v_a = v_a\right)\]
However, since we are not required to model the other auctionists, we make the following simplification:
	\[  p_a = v_a - |X|\] 
where $ X\sim \mathcal{N}(\mu,\,\sigma^{2})$ having $\mu = \frac{v_a}{10} ,\,\sigma^{2} = 0.1$,
that is assuming that we are generally paying for $\frac 9 {10}$ of our bid.
			\subsection{Daily clicks of new users}
The number of daily clicks of new users is a random variable drawn from a Gaussian distribution:
\[ X\sim \mathcal{N}(\mu,\,\sigma^{2})\]
If the task requires to find the optimal bidding strategy, we model the influence of the bid on the number of daily clicks:
\[ X\sim \mathcal{N}(\mu_0+\mu_{bid},\,\sigma^{2})\]
where $\mu_0$ is a fixed value and $\mu_{bid}$ is an increment which grows monotonically with the bid.

			\subsection{Conversion rate function}

		\section{Re-buy process}
We modeled the re-buy process as the orders the customer makes using a dash button, which comes for free together with the first bought item. We assume that the price of the additional purchases is the same of the first one.
The number of purchases (following the first one) that a customer makes in one month is modeled as a random variable with a Poisson probability distribution:
\[X \sim  \mathcal{P}oisson   (\lambda) \]
where the parameter $\lambda$ is function of the price: $\lambda(p) = \frac{\alpha}{\beta(p-k)}$, with $\alpha, \beta$ and k normalizing constants. The following diagram shows the behaviour of $X$ as the price increases:
\begin{figure}[H]
\renewcommand*\thesubfigure{\roman{subfigure}} 
\centering
\begin{subfigure}{.49\textwidth}
  \centering
  \includegraphics[width=1\linewidth]{4e}
  \caption{p = 4\euro}
  \label{fig:sub1}
\end{subfigure}
\begin{subfigure}{.49\textwidth}
  \centering
  \includegraphics[width=1\linewidth]{55e}
  \caption{p = 5.5\euro}
  \label{fig:sub2}
\end{subfigure}
\begin{subfigure}{.49\textwidth}
  \centering
  \includegraphics[width=1\linewidth]{7e}
  \caption{p = 7\euro}
  \label{fig:sub3}
\end{subfigure}
\begin{subfigure}{.49\textwidth}
  \centering
  \includegraphics[width=1\linewidth]{85e}
  \caption{p = 8.5\euro}
  \label{fig:sub4}
\end{subfigure}
	\caption{Poisson distributions generated by $\alpha=3$, $\beta=2$, $k=3.5$}
\end{figure}

\noindent The parameter $\lambda$ has different values for each class of customers. 

	%end of second chapter

	\chapter{Assignments}
		\section{Step 1}
			\subsection{Task}
\textit{Formulate the objective function when assuming that, once a user makes a purchase with a price p, then the ecommerce will propose the same price p to future visits of the same user and this user will surely buy the item. The revenue function must take into account the cost per click, while there is no budget constraint. Provide an algorithm to find the best joint bidding/pricing strategy and describe its complexity in the number of values of the bids and prices available (assume here that the values of the parameters are known). In the following steps, assume that the number of bid values are 10 as well as the number of price values.}
			\subsection{Implementation}

		\section{Step 2}
			\subsection{Task}
\textit{Consider the online learning version of the above optimization problem when the parameters are not known. Identify the random variables, potential delays in the feedback, and choose a model for each of them when a round corresponds to a single day. Consider a time horizon of one year.}
			\subsection{Implementation}

		\section{Step 3}
			\subsection{Task}
\textit{Consider the case in which the bid is fixed and learn in online fashion the best pricing strategy when the algorithm does not discriminate among the customers’ classes (and therefore the algorithm works with aggregate data). Assume that the number of daily clicks and the daily cost per click are known. Adopt both an upper-confidence bound approach and a Thompson-sampling approach and compare their performance.}
			\subsection{Implementation}

		\section{Step 4}
			\subsection{Task}
\textit{Do the same as step 3 when instead a context-generation approach is adopted to identify the classes of customers and adopt a potentially different pricing strategy per class. In doing that, evaluate the performance of the pricing strategies in the different classes only at the optimal solution (e.g., if prices that are not optimal for two customers’ classes provide different performance, you do not split the contexts). Let us remark that no discrimination of the customers’ classes is performed at the advertising level. From this step on, choose one approach between the upper-confidence bound one and the Thompson-sampling one.}
			\subsection{Implementation}
		\section{Step 5}
			\subsection{Task}
\textit{Consider the case in which the prices are fixed and learn in online fashion the best bidding strategy when the algorithm does not discriminate among the customers’ classes. Assume that the conversion probability is known. However, we need to guarantee some form of safety to avoid the play of arms that provide a negative revenue with a given probability. This can be done by estimating the probability distribution over the revenue for every arm and making an arm eligible only when the probability to have a negative revenue is not larger than a threshold (e.g., 20\%). Apply this safety constraint after 10 days to avoid that the feasible set of arms is empty, while in the first 10 days choose the arm to pull with uniform probability. Do not discriminate over the customers’ classes.}
			\subsection{Implementation}
		\section{Step 6}
			\subsection{Task}
\textit{Consider the general case in which one needs to learn the joint pricing and bidding strategy under the safety constraint introduced in step 5. Do not discriminate over the customers’ classes both for advertising and pricing.}
			\subsection{Implementation}
		\section{Step 7}
			\subsection{Task}
\textit{Do the same as step 6 when instead discriminating over the customers’ classes for pricing. In doing that, adopt the context structure already discovered in step 4.}
			\subsection{Implementation}
	%end of third chapter
	\chapter{References}
		\section{Links}

\begin{itemize}
	\item GitHub repository of the project: \url{https://github.com/tizianofucci/DIA2021AdvertisingAndPrincing}
\end{itemize}

\end{document}
